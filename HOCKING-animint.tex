\documentclass[11pt, a4paper]{article}
\usepackage{amsfonts, amsmath, hanging, hyperref, natbib, parskip, times}
\usepackage[pdftex]{graphicx}
\hypersetup{
  colorlinks,
  linkcolor=blue,
  urlcolor=blue
}

\let\section=\subsubsection
\newcommand{\pkg}[1]{{\normalfont\fontseries{b}\selectfont #1}} 
\let\proglang=\textit
\let\code=\texttt 
\renewcommand{\title}[1]{\begin{center}{\bf \LARGE #1}\end{center}}
\newcommand{\affiliations}{\footnotesize}
\newcommand{\keywords}{\paragraph{Keywords:}}

\setlength{\topmargin}{-15mm}
\setlength{\oddsidemargin}{-2mm}
\setlength{\textwidth}{165mm}
\setlength{\textheight}{250mm}

\begin{document}
\pagestyle{empty}

\title{Two new aesthetics define a grammar of interactive graphics}

\begin{center}
  {\bf Toby Dylan Hocking$^{1,^\star}$}
\end{center}

\begin{affiliations}
  1. Department of Computer Science, Tokyo Institute of Technology \\[-2pt]
  $^\star$Contact author:
  \href{mailto:toby@sg.cs.titech.ac.jp}{toby@sg.cs.titech.ac.jp}
\end{affiliations}

\keywords Interactive, Animated, Graphics, Time Series, Visualization

\vskip 0.8cm

\textbf{Motivation}: the \pkg{ggplot2} package defines a grammar that
can be used to produce many statistical graphics
\citep{ggplot2}. Although various other \proglang{R} packages can be
used to produce animated or interactive graphics, the powerful
grammar of \pkg{ggplot2} can not be used for this purpose.

The \pkg{D3} library for \proglang{Javascript} can be used to make
statistical graphics that are animated, interactive, and viewable in a
web browser \citep{D3}. However, most useRs do not have the time to
write Javascript code for everyday visualizations.

\textbf{Proposal}: define interactive animations entirely in R code by
adding 2 new aesthetics to the grammar of graphics:
\begin{itemize}
\item \code{showSelected=variable} means that only the subset of the
  data that corresponds to the selected value of \code{variable} will
  be shown.
\item \code{clickSelects=variable} means that clicking a plot element
  will change the currently selected value of \code{variable}.
\end{itemize}

In this talk I will show how an interactive animation can be defined
as a list of ggplots with these aesthetics, then automatically
translated into a D3 visualization. Demo code and examples can be
found here.

\url{http://sugiyama-www.cs.titech.ac.jp/~toby/animint/index.html}

%% references: 
\bibliographystyle{chicago}
\bibliography{refs}

%% references can alternatively be entered by hand
%\subsubsection*{References}

%\begin{hangparas}{.25in}{1}
%AuthorA (2007). Title of a web resource, \url{http://url/of/resource/}.

%AuthorC (2008a). Article example in proceedings. In \textit{useR! 2008, The R
%User Conference, (Dortmund, Germany)}, pp. 31--37.

%AuthorC (2008b). Title of an article. \textit{Journal name 6}, 13--17.
%\end{hangparas}

\end{document}
